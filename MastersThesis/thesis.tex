\documentclass[a4paper,12pt]{article}
\usepackage[T2A]{fontenc}
\usepackage[utf8]{inputenc}
\usepackage[english,russian]{babel}
\usepackage[pdftex,unicode]{hyperref}
\usepackage[pdftex]{graphicx}
\graphicspath{{figures/}}
\DeclareGraphicsExtensions{.pdf,.png,.jpg}
\usepackage{indentfirst}
\usepackage[left=2cm,right=2cm,
    top=2cm,bottom=2cm,bindingoffset=0cm]{geometry}
\usepackage{amsmath}

\title{Реферат}
\author{Елисеев}
\date{The date}
\begin{document}
	\begin{titlepage}
		\begin{center}
			\large Университет ИТМО\\[6cm]
			\huge Реферат\\[0.6cm]
			\large <<Матричная формулировка квантовой механики. Матричная алгебра.>>\\[8cm]
			\begin{flushleft}
				\emph{Студент:} Елисеев А.\\
				\emph{Группа:} V3300\\
				\emph{Преподаватель:} Комолов Владимир Леонидович\\
			\end{flushleft}
			\vfill
			\large Санкт-Петербург\\
			\large 2016
		\end{center}
		\thispagestyle{empty} % без ном. стр.
	\end{titlepage}
	\section{Оглавление}
	\tableofcontents
	\newpage
	\section{Создание квантовой механики в матричной форме}
	Первая из предложенных формулировок квантовой механики. Впервые описана в 1925 г. и развита Вернером
Гейзенбергом. В 1932 г. он за свою теорию получил нобелевскую премию "За создание квантовой механики в матричной форме". В основу теории легла идея каждой механической наблюдаемой (таким, как
положение, импульс или энергия) сопоставить матрицеу (или оператор).
	\section{Описание}
	Квантовые динамические переменные не подчиняются коммутативному закону умножения. Подобные некоммутативные динамические переменные, часто называются просто операторами, их удобно представлять в виде матриц. Обычно, в квантовой механике имеют дело с эрмитовыми и унитарными матрицами бесконечного ранга. Например матрицу оператора энергии с помощью некоторого полного набора ортонормированных функций $\nu_{n}(r)$ можно представить в седующем виде:
	\begin{equation}
		H_{nm}=\int d\tau \nu ^{*}_{n}(r)\hat{H} \nu_{m}(r)  
	\end{equation}
	Такая матрица позволяет определить энергетические состояния системы. Аналогично, в матричном виде записываются и другие наблюдаемые величины.
	\section{Матричная алгебра}
	\subsection{Определенее}
	Для начала необходимо вести определение матрицы. Матрица - это математический объект, записываемый в виде прямоугольной таблицы элементов кольца или поля, которая представляет собой совокупность строк и столбцов, на пересечении которых находятся её элементы. Количество строк и столбцов матрицы задает размер матрицы. Такая таблица складывается и перемножается с другими таблицами по определенным формулам.\par
	Рассмотрим например матрицы $A$ $n$ на $m$ и $B$ $k$ на $s$
\begin{equation}
A = \begin{pmatrix}
a_{11} & a_{12} & \cdots & a_{1m} \\
a_{21} & a_{22} & \cdots & a_{2m} \\         
\vdots & \vdots & \ddots & \vdots \\
a_{n1} & a_{n2} & \cdots & a_{nm}
\end{pmatrix}
\end{equation}
\begin{equation}
B = \begin{pmatrix}
b_{11} & b_{12} & \cdots & b_{1s} \\
b_{21} & b_{22} & \cdots & b_{2s} \\         
\vdots & \vdots & \ddots & \vdots \\
b_{k1} & b_{k2} & \cdots & b_{ks}
\end{pmatrix}
\end{equation}
\subsection{Сложение}
Матрицы $A$ $B$ можно сложить, если $n=k$ и $m=s$, тогда элементы матрицы $C=A+B$ вычисляется следующим образом:
\begin{equation}
c_{i,j}=a_{i,j}+b_{i,j}
\end{equation}
Здесь $i = \{1,2\ldots n\}$ и $j = \{1,2\ldots m\}$
\subsection{Умножение}
Если число столбцов матрицы $A$ равно числу строк матрицы $B$ ($m=k$), то матрицы можно перемножить ($A$ умножить на $B$ справа), по следующему правилу: если $C=AB$ то:
\begin{equation}
c_{i,j}=\Sigma_{t=1}^{m}a_{i,t}b_{t,j}
\end{equation}
\subsection{Свойства умножения и сложения}
\begin{itemize}
\item Сложение матриц коммутативно $A+B=B+A$
\item Умножение матриц дистрибутивно $A(B+C)=AB+AC$
\item Умножение матриц ассоциативно $A(B+C)=AB+AC$
\end{itemize}
\subsection{Нулевая матрица}
Матрица, все элементы которой равны нулю. Для квадратной нулевой матрицы и квадратной матрицы $A$ выполняется $A0=0A$.
\subsection{Единичная матрица}
Квадратная матрица, на главной диагонали которой стоят единичные матричные элементы.
\begin{equation}
E = \begin{pmatrix}
1 & 0 & \cdots & 0 \\
0 & 1 & \cdots & 0 \\         
\vdots & \vdots & \ddots & \vdots \\
0 & 0 & \cdots & 1
\end{pmatrix}
\end{equation}
Для единичной матрицы справедливо $AE=EA$
\subsection{Произведение матрицы на число}
Произведением числа $q$ на матрицу $A$ является матрица, элементы которой получаются умножением матричных элементов матрицы $A$ на это число.
\begin{equation}
q A = \begin{pmatrix}
q a_{11} & q a_{12} & \cdots & q a_{1m} \\
q a_{21} & q a_{22} & \cdots & q a_{2m} \\         
\vdots & \vdots & \ddots & \vdots \\
q a_{n1} & q a_{n2} & \cdots & q a_{nm}
\end{pmatrix}
\end{equation}
\subsection{Шпур}
Сумма диагональных матричных элементов квадратной матрицы называется ее шпуром или следом (обозначается как Spur, Trace)
\begin{equation}
	Sp(A)=\Sigma_{k=1}^{n} A_{k,k}
\end{equation}
\subsection{Обратная матрица}
Матрица $A$ может иметь или не иметь обратную матрицу $A^{-1}$, определяемую соотношениями
\begin{equation}
A A^{-1}=A^{-1} A
\end{equation}
Если матрица A имеет обратную, то она называется несингулярной и является квадратной.
\subsection{Эрмитово сопряжение}
Матрица $A^+$  называется эрмитово сопряженной с $A$, если она получена из $A$ заменой строк на столбцы и всех матричных элементов на комплексно сопряженные им величины. Это можно записать следующим образом.
\begin{equation}
A = \begin{pmatrix}
a_{11} & a_{12} & \cdots & a_{1m} \\
a_{21} & a_{22} & \cdots & a_{2m} \\         
\vdots & \vdots & \ddots & \vdots \\
a_{n1} & a_{n2} & \cdots & a_{nm}
\end{pmatrix}
\end{equation}
\begin{equation}
A^+ = \begin{pmatrix}
a^*_{11} & a^*_{12} & \cdots & a^*_{1m} \\
a^*_{21} & a^*_{22} & \cdots & a^*_{2m} \\         
\vdots & \vdots & \ddots & \vdots \\
a^*_{n1} & a^*_{n2} & \cdots & a^*_{nm}
\end{pmatrix}
\end{equation}
\begin{equation}
(A^+)_{kl}=A^*_{lk}
\end{equation}
\subsection{Самосопряженность и унитарность}
\begin{itemize}
\item Матрица называется эрмитовой или самосопряженной, если $A=A^+$
\item Унитарность матрицы заается следующим образом $A^+=A^{-1}$
\end{itemize}
\subsection{Диаганализация}
Матрица $S$ диагонализует матрицу $A$, если в результате преобразования получается диагональная матрица $A'$
\begin{equation}
A'_{kl}=A'_k \delta_{kl}
\end{equation}
Собственные значения диагональной матрицы $A'$, полученной из $A$ в результате преобразования, не зависят от способа диагонализации $A$: они являются и собственными значениями первоначальной недиагональной матрицы $A$.
\subsection{Важные свойства в квантовой механике}
Основная теорема: любую эрмитову матрицу можно привести к диагональному виду с помощью унитарного преобразования.\par
Следствие: собственные значения эрмитовой матрицы определяются однозначно.\par
Чтобы две эрмитовы матрицы можно было диагонализовать с помощью одного и того же унитарного преобразования, необходимо и достаточно чтобы эти матрицы коммутировали, т.е.
\begin{equation}
AB=BA
\end{equation}
Или через оператор коммутации $[AB]=0$ \par
Собственные значения эрмитовой матрицы вещественны.
\end{document}



















