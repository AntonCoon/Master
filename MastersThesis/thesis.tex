\documentclass[a4paper,14pt,russian]{extreport}

%Чтобы работали все шрифты
\usepackage{extsizes}
%\usepackage{cmap} % для кодировки шрифтов в pdf
\usepackage[T2A]{fontenc}
\usepackage[utf8]{inputenc}
\usepackage[russian]{babel}

%Для рисуночков
\usepackage[pdftex]{graphicx}
\graphicspath{{figures/}}
\DeclareGraphicsExtensions{.pdf,.png,.jpg}
\usepackage{amssymb,amsfonts,amsmath,amsthm}
\usepackage{indentfirst}
\usepackage[usenames,dvipsnames]{color}
\usepackage{makecell}
\usepackage{multirow}

%Настройка заголовков
\usepackage{titlesec}
 \titleformat{\chapter}[display]
    {\filcenter}
    {\MakeUppercase{\chaptertitlename} \thechapter}
    {8pt}
    {\bfseries}{}
 \titleformat{\section}
    {\normalsize\bfseries}
    {\thesection}
    {1em}{}
 \titleformat{\subsection}
    {\normalsize\bfseries}
    {\thesubsection}
    {1em}{}
 % Настройка вертикальных и горизонтальных отступов
\titlespacing*{\chapter}{0pt}{-30pt}{8pt}
\titlespacing*{\section}{\parindent}{*4}{*4}
\titlespacing*{\subsection}{\parindent}{*4}{*4}


 
\linespread{1.3}
\renewcommand{\rmdefault}{ftm}
\frenchspacing


\usepackage{tocloft}
\renewcommand{\cfttoctitlefont}{\hspace{0.38\textwidth} \bfseries\MakeUppercase}
\renewcommand{\cftbeforetoctitleskip}{-1em}
\renewcommand{\cftaftertoctitle}{\mbox{}\hfill \\ \mbox{}\hfill{\footnotesize Стр.}\vspace{-2.5em}}
\renewcommand{\cftchapfont}{\normalsize\bfseries \MakeUppercase{\chaptername} }
\renewcommand{\cftsecfont}{\hspace{31pt}}
\renewcommand{\cftsubsecfont}{\hspace{11pt}}
\renewcommand{\cftbeforechapskip}{1em}
\renewcommand{\cftparskip}{-1mm}
\renewcommand{\cftdotsep}{1}
\setcounter{tocdepth}{2} % задать глубину оглавления — до subsection включительно

\usepackage[tableposition=top]{caption}
\usepackage{subcaption}
\DeclareCaptionLabelFormat{gostfigure}{Рисунок #2}
\DeclareCaptionLabelFormat{gosttable}{Таблица #2}
\DeclareCaptionLabelSeparator{gost}{~---~}
\captionsetup{labelsep=gost}
\captionsetup[figure]{labelformat=gostfigure}
\captionsetup[table]{labelformat=gosttable}
\renewcommand{\thesubfigure}{\asbuk{subfigure}}

\usepackage{fancyhdr}
\pagestyle{fancy}
\fancyhf{}
\fancyhead[C]{\thepage}
\renewcommand{\headrulewidth}{0pt}


\usepackage{geometry}
\geometry{left=2.5cm}
\geometry{right=1cm}
\geometry{top=2cm}
\geometry{bottom=2cm}

\newcommand{\empline}{\mbox{}\newline}
\newcommand{\likechapterheading}[1]{ 
    \begin{center}
    \textbf{\MakeUppercase{#1}}
    \end{center}
    \empline}

\makeatletter
    \renewcommand{\@dotsep}{2}
    \newcommand{\l@likechapter}[2]{{\bfseries\@dottedtocline{0}{0pt}{0pt}{#1}{#2}}}
\makeatother
\newcommand{\likechapter}[1]{    
    \likechapterheading{#1}    
    \addcontentsline{toc}{likechapter}{\MakeUppercase{#1}}}


\title{Бакалавр}
\author{Елисеев}
\date{The date}
\begin{document}
	\def\contentsname{СОДЕРЖАНИЕ}
	\begin{titlepage}
		\begin{center}
			\large Университет ИТМО\\[2cm]
			\huge Мой прекрасный диплом\\
			\large <<СВЕРХБЫСТРАЯ ДИНАМИКА НОСИТЕЛЕЙ ЗАРЯДА В ПОЛУПРОВОДНИКОВЫХ НИТЕВИДНЫХ НАНОКРИСТАЛЛАХ.>>\\[3cm]
			\begin{flushleft}
				\emph{Студент:} Елисеев А.\\
				\emph{Группа:} V3400\\
				\emph{Научрук:} Валерий Николаевич\\
			\end{flushleft}
			\vfill
			\large Санкт-Петербург\\
			\large 2017
		\end{center}
		\thispagestyle{empty} % без ном. стр.
	\end{titlepage}
	\setcounter{page}{2}
	\likechapter{Аннотация}
		\newpage
	\tableofcontents
	\chapter{Введение}
		Полупроводниковые наноструктуры в виде свободно стоящих
полупроводниковых нитевидных нанокристаллов (ННК) являются одним из
наиболее перспективных нанообъектов для применения в наноэлектронике,
нанофотонике и нанобиоэлектронике. ННК используются для создания
сверхчувствительных фотодиодов [статья про это], транзисторов сверхвысокой плотности
[статья про это], эмиттеров излучения видимого диапазона волн [статья про это] и ТГц диапазона [статья В.Н.].\par
Решающую роль 
		\section{Использование ННК в качестве эмиттеров в ТГц спектроскопии}
			Найти первую статью про ТГц спектроскопию.\par
			Рассказать о том, почему лучше использовать ННК в качестве эмиттеров ТГц.\par
			Сослаться на статью, про то, что в полупроводниковых ННК ТГц генерится за счет движения носителей.
		\section{Динамика носителей в ННК}
			Какие есть работы и что в них изучено.\par
			Чего нет и почему это необходимо.
	\chapter{Основная часть}
			Коротко о том, что я напишу в этой главе.
		\section{Зависимость ТГц излучения от динамики}
			Коротко, о том, от чего зависит ТГц излучение от ННК. Определяющие процессы.	
		\section{Схема установки, описание метода}
			Ссылочка На статью, где впервые описан этот метод и его описание\par	
			Схема, ссылка на приложение, в котором описаны характеристики элементов, используемых в схеме.\par
		\section{Упорядоченные образцы ННК GaAs}
			Метод газофазной эпитаксии, ссылка на статью и короткое описание с картиночкой\par
			Ориентация $GaAs$, получившиеся образцы, фото СЭМ
		\section{Полученные результаты}
			Типичные волновые формы\par
			Динамика, для упорядоченных образцов, при разной мощности накачки\par
			Характерные участки (короткая и длинная динамика) \par
			Зависимость от мощности накачки, для короткой динамики.\par
			Объяснение результатов, гипотезы, предположения.\par
			Возможно, спектральные компоненты, для подтверждения предположений
		\section{Упорядоченные образцы с шубой}
			Узнать у ВН и разобраться самому
		\section{Связь GaAs и AlGaAs}
			Наверняка в динамике должно быть видно проявление изменения концентрации ловушек на поверхности и вообще 					изменения встроенного поля. Тут же надо привести зонную диаграмму.
		\section{Неупорядоченные ННК на основе GaAs}
			Метод получения, ссылка на статью и короткое описание.
		\section{Динамика в неупорядоченных ННК}
			Динамика, основные параметры
		\section{Анализ и сравнение для разных образцов}
			Объяснение разницы в динамике
		\section{Те самые образцы, для которых эффективность генерации увеличивается}
			Процессы, отвечающие за генерацию ТГц в этих образцах.\par
			Почему таки происходит увеличение эффективности.\par
			Сравнить "наилучшую" эффективность для каждого из образцов
	\chapter{Заключение}
		\section{Динамика}
			Все, что удалось узнать.
		\section{Где следует применить полученные результаты}
			Наверное важно сказать об этом.
		\section{Положения дипломной работы}
			Все что удалось узнать, но в виде выражений и емких утверждений.
			\newpage
	\likechapter{Список сокращений и условных обозначений}
		\newpage
	\likechapter{Список терминов}
		\newpage
	\likechapter{Список использованных источников}
		\newpage
	\likechapter{Приложения}
\end{document}



















